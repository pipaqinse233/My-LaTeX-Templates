\documentclass{qmes-template}

\title{Report Template of QMES}
\course{QXU114514}
\date{\today}

\wordcount{1919810}

\addstudent{Zhang San}{20231234}
\addstudent{Li Si}{20245678}

\begin{document}

\maketitle
\tableofcontents

\section{Introduction}
This template is design for student of QMUL engineering school, NPU.

The main content of the document begins here. And page numbering starts here.

\section{Methodology}
Sample text for demonstrating the template features.

\subsection{User's customization}
For better use, the template has some customizable content.There is some
basically customization below:
\begin{lstlisting}
\title{Edit this to your report name}
\course{The course code}
\end{lstlisting}

Generally speaking, no changes are needed in this line. Unless you need the
report to show a date that's not today.
\begin{lstlisting}
\date{\today}
\end{lstlisting}

For the security reasons, users can only count the number of characters by
themselves. it's recommended that count your PDF files.
\begin{lstlisting}
\wordcount{The word count of your report}
\end{lstlisting}

For each team member, you need to add their personal information with a single
addstudent directive so that they can be displayed in the author column of the
title page
\begin{lstlisting}
\addstudent{your team parter's name}{his/her student number}
\end{lstlisting}

\subsection{Listings}
Example of implementing code blocks throught listings is below:

\includecode[caption={The Implementing of Code Block Below}]{src/main.tex}

\begin{lstlisting}[language=Python,caption={Example Code 2}] 
    def hello_world():
        print("Python code example") 
\end{lstlisting}

Here shows implement code blocks throught include code file:
\begin{lstlisting}[caption={The Implementing of Code Block Below}]
\includecode[language=C++,caption={Example for include code file}]{scr/main.cpp}
\end{lstlisting}
% Include external code file
\includecode[language=C++,caption={Example for include code file}]{src/main.cpp}

\end{document}
